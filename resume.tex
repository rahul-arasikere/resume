%%%%%%%%%%%%%%%%%%%%%%%%%%%%%%%%%%%%%%%%%
% Developer CV
% LaTeX Template
% Version 1.0 (28/1/19)
%
% This template originates from:
% http://www.LaTeXTemplates.com
%
% Authors: Rahul Arasikere
% Jan Vorisek (jan@vorisek.me)
% Based on a template by Jan Küster (info@jankuester.com)
% Modified for LaTeX Templates by Vel (vel@LaTeXTemplates.com)
%
% License:
% The MIT License (see included LICENSE file)
%
%%%%%%%%%%%%%%%%%%%%%%%%%%%%%%%%%%%%%%%%%

%----------------------------------------------------------------------------------------
%	PACKAGES AND OTHER DOCUMENT CONFIGURATIONS
%----------------------------------------------------------------------------------------

\documentclass[9pt]{developercv} % Default font size, values from 8-12pt are recommended

%----------------------------------------------------------------------------------------

\begin{document}

%----------------------------------------------------------------------------------------
%	TITLE AND CONTACT INFORMATION
%----------------------------------------------------------------------------------------

\begin{minipage}[t]{0.365\textwidth} % 45% of the page width for name
  \vspace{-\baselineskip} % Required for vertically aligning minipages

  % If your name is very short, use just one of the lines below
  % If your name is very long, reduce the font size or make the minipage wider and reduce the others proportionately
  {\HUGE\textbf{\MakeUppercase{Rahul}}} % First name

  \vspace{6pt}

  {\HUGE\textbf{\MakeUppercase{Arasikere}}} % First name

  \vspace{6pt}

  \colorbox{RoyalBlue}{{\huge\textcolor{white}{Software Developer}}} % Career or current job title
\end{minipage}
\begin{minipage}[t]{0.335\textwidth} % 27.5% of the page width for the first row of icons
  \vspace{-\baselineskip} % Required for vertically aligning minipages

  % The first parameter is the FontAwesome icon name, the second is the box size and the third is the text
  % Other icons can be found by referring to fontawesome.pdf (supplied with the template) and using the word after \fa in the command for the icon you want
  \icon{Linkedin}{12}{\href{https://linkedin.com/in/rahul-arasikere}{linkedin.com/in/rahul-arasikere}}\\
  \icon{Phone}{12}{617-708-7291}\\
  \icon{At}{12}{\href{mailto:arasikere.rahul@gmail.com}{arasikere.rahul@gmail.com}}\\
\end{minipage}
\begin{minipage}[t]{0.3\textwidth} % 27.5% of the page width for the second row of icons
  \vspace{-\baselineskip} % Required for vertically aligning minipages

  % The first parameter is the FontAwesome icon name, the second is the box size and the third is the text
  % Other icons can be found by referring to fontawesome.pdf (supplied with the template) and using the word after \fa in the command for the icon you want
  \icon{Globe}{12}{\href{https://rahul-arasikere.github.io/?ref=resume}{rahul-arasikere.github.io}}\\
  \icon{Github}{12}{\href{https://github.com/rahul-arasikere}{github.com/rahul-arasikere}}\\
  \icon{Twitter}{12}{\href{https://twitter.com/@rahul_arasikere}{@rahul\_arasikere}}\\
\end{minipage}

\vspace{0.5cm}

%----------------------------------------------------------------------------------------
%	EDUCATION
%----------------------------------------------------------------------------------------

\cvsect{Education}

\begin{entrylist}
  \entry
  {9/2018 -- 5/2022\\\footnotesize{ongoing}}
  {Bachelor's Degree in Computer Science}
  {Boston University}
  {CGPA: 3.14/4.0.}
\end{entrylist}

%----------------------------------------------------------------------------------------
%	EXPERIENCE
%----------------------------------------------------------------------------------------

\cvsect{Experience}

\begin{entrylist}
  \entry
  {5/2021 -- present}
  {Software Engineering Intern}
  {Shell Techworks}
  {Working on developing a real time hydrogen refueling station as part of the renewable energy commitment by shell.\\
    \texttt{FreeRTOS}\slashsep\texttt{Embedded - ARM}\slashsep\texttt{C}}
  \entry
  {6/2020 -- 8/2020}
  {Data Science Intern}
  {Simplify360}
  {Worked on automated text tagging and classification based on the clients' requirements to interface with their B2B systems. \\
    \texttt{NLP}\slashsep\texttt{Pytorch}}
  \entry
  {6/2019 -- 8/2019}
  {Data Science Intern}
  {Remidio Innovative Solutions Pvt. Ltd.}
  {Developed an automated pipeline for detection of age-related macular degeneration in fundus images captured from Remidio’s telematic platform. Model achieved 82\% accuracy over 3200 images of various qualities. The model was based on Resnet18.\\ \texttt{Python}
    \slashsep\texttt{Keras}\slashsep\texttt{Pytorch}}
  \entry
  {6/2018 -- 7/2018}
  {Intern}
  {Spectral Insights}
  {Worked with a team of 2 – 3 people in implement AI on top of digital images of renal biopsies to automatically detect glomeruli and display it on the screen to the doctor. Gained technical insight and experience as to how professional companies’ function. \\ \texttt{C++}\slashsep\texttt{OpenCV}
    \slashsep\texttt{Linux}}
  \end{entrylist}

%----------------------------------------------------------------------------------------
%	ADDITIONAL INFORMATION
%----------------------------------------------------------------------------------------
\cvsect{Projects}
\begin{entrylist}
  \entry
  {3/2021 -- 5/2021}
  {Student}
  {BU Spark!}
  {Researched the feasibility of utilizing Linux Industrial IO subsystem on Intel processors.\\
  \texttt{C}\slashsep\texttt{Linux}}
  \entry
  {7/2020 -- 12/2020}
  {Student}
  {}
  {Developed an interpreter as well as a C code generator for lambda calculus in ATS.}
  \entry
  {5/2020 -- 8/2020}
  {Software Developer}
  {XRTerra AR VR Developer Cohort}
  {Worked within a team of 5 to develop various AR and VR projects for clients such as PTC
    and Packet39. Developed a crane training VR simulator and note taking app that utilized
    spatial anchors, allowing engineers to tag virtual notes in real life.\\
    \texttt{Unity}\slashsep\texttt{Unity MARS}\slashsep\texttt{C\#}\slashsep\texttt{Azure}}
  \entry
  {1/2020 -- 5/2020}
  {Software Developer}
  {BU SPARK! Fellowship Program}
  {Worked with a team of 7, to develop a tab management system for students and
    scholars to research more productively online. 100\% of users felt an increase in
    their productivity and motivation during user research and surveys. Awarded Best
    Idea and Best Design on SPARK Demo day.\\
    \texttt{JavaScript}\slashsep\texttt{React}\slashsep\texttt{Firebase}}
  \entry
  {9/2018 -- Present}
  {Software Developer}
  {Boston University Rocket Propulsion Group}
  {Lead and developed a concurrent socket based data acquisition software for custom test stands.
    Developing RTOS based flight computer firmware for sub-orbital launch vehicle. Worked
    on telemetry modules using LoRa. Maintained legacy code bases written in Java.\\\texttt{Arduino}\slashsep\texttt{C}\slashsep\texttt{Python}
    \slashsep\texttt{FreeRTOS}\slashsep\texttt{Java}}

\end{entrylist}
%----------------------------------------------------------------------------------------

\end{document}
